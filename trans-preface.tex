%% Copyright (c) 2014 Li Fanxi <lifanxi@freemindworld.com>
%% Permission is granted to copy, distribute and/or modify this
%% document under the terms of the GNU Free Documentation License,
%% Version 1.3 or any later version published by the Free Software
%% Foundation; with no Invariant Sections, no Front-Cover Texts, and
%% no Back-Cover Texts. A copy of the license is included in the
%% file called ``gfdl.tex''.
\chapter{译者序}

1999年,我第一次尝试安装GNU/Linux,这也应该是我第一次接触``自由软件''。这次尝试的结果就是让我损失了硬盘上积累的自从1994年第一次接触电脑以来所有的资料。虽然遭遇了这样的不幸,却仍然磨灭不了我的好奇心。1999年到2006年期间,我屡战屡败地安装过各种各样的GNU/Linux发行版,并最终决定用GNU/Linux作为我日常使用的操作系统。

2004年,我第一次接触Emacs编辑器。Emacs实在是个难以上手的家伙,好不容易通过Emacs内置的教程学到一点皮毛后,我就决定尝试把这份教程翻译成中文,造福后人。我向Emacs的邮件列表发了一封热情洋溢的邮件,咨询是否可以由我来完成翻译工作。很快收到了两封回信,说CVS中已经有人完成了教程的中文翻译,会随着下一个版本的Emacs发布,并欢迎我对现有的译文进行审阅。

在继续学习Emacs的过程中,我终于认识了GNU工程,也才真正明白了什么是``自由软件''。后来我加入了GNU网站的中文翻译小组,在翻译GNU网站的各种页面和文章的过程中,对自由软件、GPL以及自由软件运动发起人理查德·斯托曼有了进一步的认识。

2008年5月24日,我参加了在上海复旦大学举行的``哲思自由软件峰会'',第一次与斯托曼有了面对面的交流。除了他所使用的OLPC XO笔记本电脑让我垂涎欲滴外,他在现场所表现出来的强烈个性更是给我留下了深刻的印象。会后,我让他在某本图书上签名的请求被他毫不留情的拒绝了,``这不是一本自由图书,我不能在上面签名。''斯托曼的解释非常干脆。

2011年的某一天,我在无意中发现,七年前回复我有关Emacs教程翻译问题的人之一就是斯托曼本人。我与斯托曼的这两次直接交流,得到了截然不同的反馈:对于向自由软件社区贡献自己力量的行为,斯托曼会给予热情的支持和帮助;对于非自由的软件和图书作品,斯托曼的抵触毫不留情。

在GNU网站的中文翻译小组中,我和邓楠一度是比较活跃的两位成员。作为翻译小组的协调员,邓楠需要负责所有翻译成果的发布工作。我们的合作非常愉快,在2010年左右完成了GNU网站大部分二级页面的翻译工作。

所以,当邓楠邀请我跟他一起完成本书的翻译工作时,我马上就答应了。虽然这只是一本薄薄的小册子,但是对于我这样习惯于技术言语的人来说,要把这么多语句组织得不是那么干涩还是有着不小的挑战。中间的曲折故事暂且不表,经过近一年的努力,终于我们还是完成了这本书的翻译工作。

感谢人民邮电出版社陈冀康编辑在本书出版过程给予的帮助,正如本书作者在跋中所写的那样,在传统出版领域出版这样一本``自由''图书还是面临着不少的挑战。陈冀康编辑的耐心和执着,最终促成了本书。感谢好友贾征主动帮忙审阅了全书的内容,也感谢家人对我翻译工作的支持。

尤其值得一提的是,这本书使用的是GFDL许可证,也就是说,这是市面上极为少有的一本``自由''图书。在GFDL的条款的保护下,读者可以自由的复制、分发和修改本书。我们已经把全书的\LaTeX源文件公开发布到网上,希望通过我们的努力,抛砖引玉,让更多的人可以参与到了解、完善和传播本书内容的活动中。请访问\url{http://www.freemindworld.com/faif}获取本书的电子版本,并欢迎读者参与到本书的修订工作中来。下回如果你有机会见到斯托曼本人,不妨带上本书,他一定会非常乐意在这本记录他人生故事的``自由图书''上签名的。


李凡希\\
2014年3月于杭州
